%ltex: language=de
\chapter{Simulation}
	Um das Verhalten des Spannungsreglers vom Typ LM317 zu untersuchen wird die im Kapitel 9.2 \textit{Typical Application} des Datenblattes
	vorgeschlagene Schaltung in \textsc{LTSpice} aufgebaut und simuliert. Anschließend werden jeweils die simulierten Werte mit den
	rechnerisch zu erwartenden Werten verglichen.
	\section{Leerlaufspannung}\label{sec:leerlaufspannung}
		Nach \cref{tab:lm317 characteristics} wird eine Mindestlast von \SI{3,5}{mA} bei einer Differenzspannung zwischen Eingang
		und Ausgang des Spannungsreglers von \SI{40}{V} benötigt, um die gewünschte Ausgangsspannung aufrecht halten zu können.
		Um auch bei offenen Ausgangsklemmen eine regulierte Ausgangsspannung sicherzustellen, kann das Widerstandspaar \(R_1\) und \(R_2\)
		so dimensioniert werden, dass sich der gewünschte Ausgangsstrom einstellt. Bereits mit den in \cref{eq:ausgangsspannung voila} gewählten
		Werten wird eine konstante Last von \SI{4,6}{mA} sichergestellt. Hier stellt sich eine Ausgangsspannung von \(U_{out} = \SI{5,07525}{V}\) ein.\par
		\Cref{fig:plot minimum load} zeigt den simulierten Verlauf der Ausgangsspannung als Funktion des Widerstandes \(R_1\) im Intervall
		von \SI{240}{\ohm} bis \SI{1500}{\ohm}.
		%
		\begin{figure}[h]
			\centering
			\includesvg[inkscapelatex=false, width=.9\textwidth]{LTSpice/plots_schematics/dia_minimum_load}
			\caption[Simulation der Ausgangsspannung im Leerlauf in Abhängigkeit von \(R_1\)]{Simulation der Ausgangsspannung \(U_{out}\) als Funktion des Widerstandes \(R_1\) mit der Eingangsspannung \(U_{in}\) als Parameter.
			Die steilere Kurve zeigt den Verlauf bei einer Eingangsspannung von \SI{45}{V}. Die flachere Kurve den, bei einer Eingangsspannung von \SI{12}{V}.}
			\label{fig:plot minimum load}
		\end{figure}
		%
		Der steilere Verlauf entspricht einer Eingangsspannung von \SI{45}{V}. Gemäß Datenblatt kann bei einem typischen Laststrom von \(< \SI{3.5}{mA}\) mit einer Spannungsdifferenz von
		\SI{40}{V} zwischen Ausgang und Eingang keine Regelung mehr garantiert werden. Die Beuge bei \(\approx \SI{340}{\ohm}\) entspricht einem
		Ausgangsstrom von
		%
		\begin{align}
			I_{out} &= \frac{U_{out}}{R_1 + R_1 \cdot 3,037} = \frac{\SI{5}{V}}{\SI{340}{\ohm} + \SI{340}{\ohm} \cdot 3,037} \nonumber \\
					&\approx \SI{3,6}{mA}
			\label{eq:ausgangsstrom bei 45V in und ohne last}
		\end{align}
		%
		was sich recht gut mit den Angaben im Datenblatt deckt.\par
		Die flachere Kurve hingegen wurde mit einer Spannungsdifferenz von \SI{12}{V} simuliert. Hier bricht die Spannungsregelung
		deutlich später bei \(\approx \SI{940}{\ohm}\) ein, was einem Laststrom von \(\approx \SI{1,3}{mA}\) entspricht.
		%
	\section{Strom Adjust-Anschluss}
		Mit den Werten für \(R_1 = \SI{270}{\ohm}\) und \(R_2 = \SI{820}{\ohm}\) was einem Laststrom von \(\approx \SI{4.6}{mA}\) entspricht,
		stellt sich ein Strom aus dem Adjust-Anschluss von \(\approx \SI{47.5}{\micro A}\) ein.
		%
	\section{Referenzspannung}
		%
		Vom Hersteller wird angegeben, dass die Referenzspannung zwischen dem Ausgang des LM317 und seinem Adjust-Anschluss gemessen werden kann und nominell
		\SI{1,25}{V} beträgt (vgl. \cref{fig:typical app sch}).\par
		\textsc{LTSpice} lässt jedoch nur Messung gegen GND zu. Um die Referenzspannung zu finden wird also die Spannung am Adjustanschluss von der Ausgnagsspannung
		subtrahiert um die Referenzspannung zu erhalten. Die Simulation gibt hier einen Wert von
		\begin{equation}
			U_{Ref} = U_{out} - U_{Adj} = \SI{5,07525}{V} - \SI{3,8378}{V} \approx \SI{1,23745}{V}
		\end{equation}
	\section{Maximaler Ausgangsstrom}
		Dem Datenblatt folgend ist der Linearregler -- abhängig von der Spannungsdifferenz zwischen Ein- und Ausgang -- in seinem maximalen
		Ausgangsstrom limitiert. Simulationsergebnisse im Vergleich mit den Angaben im Datenblatt sind in \cref{tab:maximaler strom} aufgeführt.
		\begin{table}[h]
			\centering
			\caption[Gegenüberstellung von \(I_{out,max}\) und den Angaben im Datenblatt]{Vergleich der im Datenblatt angegebenen Werte für den maximalen Ausgangsstrom mit den simulierten Werten.}
			\begin{tabular}{@{}SSS@{}}
				\toprule
				\multicolumn{1}{l}{Eingangsspannung}	&\multicolumn{2}{c}{Ausgangsstrom}		\\
				\multicolumn{1}{c}{\(U_{in}\) / \(V\)}				&\multicolumn{1}{c}{\(I_{out,typ}\) / \(A\)}&\multicolumn{1}{c}{\(I_{out,sim}\) / \(A\)}\\
				\midrule
				12					&2,2		&2,15615\\
				45					&0,4		&0,788137\\
				\bottomrule				
			\end{tabular}
			\label{tab:maximaler strom}
		\end{table}
		%
	\section{Temperaturstabilität der Ausgangsspannung}
		Kapitel 7.5 des Datenblattes zum LM317 ist eine Ausgangsspannungsstabilität von typischerweise \SI{0,7}{\percent} im Temperaturbereich
		von \SI{0}{\celsius} bis \SI{125}{\celsius} zu entnehmen.
		\begin{figure}[h]
			\centering
			\includesvg[inkscapelatex=false, width=.9\textwidth, height=8cm]{LTSpice/plots_schematics/temp_stab.svg}
			\caption[Verlauf der Ausgangsspannung im Temperaturbereich \SI{0}{\celsius} bis \SI{125}{\celsius}]{Plot des simulierten Spannungsverlaufs der Ausgangsspannung im Temperaturbereich von \SI{0}{\celsius} bis \SI{125}{\celsius}.}
			\label{fig:temp stab}
		\end{figure}
		Mit Blick auf \cref{fig:temp stab} offenbart eine Simulation obiger Schaltung
		im angegebenen Temperaturbereich eine Abweichung der Ausgangsspannung von \SI{+0,02132}{V} relativ zur Ausgangsspannung bei Raumtemperatur von
		\SI{5,07525}{V}. Dies entspricht einem relativen Anstieg von \SI{0,42}{\percent}.\par
		Durch den starl exponentiellen Verlauf der Ausgangsspannung bei steigender Temperatur erscheint es nicht sinnvoll, einen
		linearen Temperaturkoeffizienten anzunehmen.
		%
	\section{Verhalten bei Eingangsspannungssprüngen}
		Um das Verhalten der Schaltung bei sprunghaften Änderungen der Eingangsspannung zu untersuchen wurde der Eingangsspannung ein symmetrisches Rechtecksignal
		mit \({U_pp} = \SI{2}{V}\), \(t_{rise} = t_{fall} = \SI{1}{\micro \second}\) und einer Periodendauer von \(T = \SI{100}{m \second}\) hinzuaddiert (vgl. \cref{fig:pulsefigs}).
		Die Simulation wurde mit einem Wert für \(C_1 = \SI{10}{\micro F}\) durchgeführt.\par\medskip
		\Cref{subfig:sag} zeigt einen Zoom auf den Verlauf der Ausgangsspannung bei fallender Flanke des Eingangssignals.
		%
		\begin{figure}[h]
			\centering
			\begin{subfigure}[t]{\textwidth}
				\centering
				\includesvg[inkscapelatex=false, width=\textwidth]{LTSpice/plots_schematics/pulse}
				\caption{}
				\label{subfig:pulse}
			\end{subfigure}
			\begin{subfigure}[l]{.45\textwidth}
				\centering
				\includesvg[inkscapelatex=false, width=\textwidth]{LTSpice/plots_schematics/pulse_output_sag}
				\caption{Reaktion bei fallender Flanke.}
				\label{subfig:sag}
			\end{subfigure}
			\hfill
			\begin{subfigure}[r]{.45\textwidth}
				\centering
				\includesvg[inkscapelatex=false, width=\textwidth]{LTSpice/plots_schematics/pulse_output_spike}
				\caption{Reaktion bei steigender Flanke.}
				\label{subfig:spike}
			\end{subfigure}
			\caption[Puls-Reaktion der Ausgangsspannung bei sprunghaften Änderungen der Eingangsspannung]{Pulse-Reaktion der Ausgangsspannung (grün) bei sprunghaften Änderungen der Eingangsspannung (blau).}
			\label{fig:pulsefigs}
		\end{figure}
		%
		Es ist zu erkennen, dass \(U_{out}\) um \(\approx \SI{84}{mV}\) einbricht mit einer Einschwingzeit zurück zur nominellen Ausgangsspannung von \SI{0,3}{ms}.
		In \cref{subfig:spike} wird die Reaktion auf eine steigende Flanke des der Eingangsspannung überlagerten Rechtecksignals dargestellt.
		Hier beträgt der größte Ausschlag \(\approx \SI{320}{mV}\) mit einer Einschwingdauer von etwa \(\SI{240}{\micro s}\).\par\medskip
		\Cref{fig:line transient response} aus Kapitel 9.2.3 \textit{Application Curves} des Datenblattes zeigt die Reaktion auf ein Rechtecksignal mit \(U_{pp} = \SI{1}{V}\)
		und einer Ausgangsspannung von etwa \SI{10,06}{V}. Hier wird ein Überschwingen auf \SI{10,1}{V} mit einer Dauer von \(> \SI{3}{\micro \second}\)
		und ein Unterschwingen auf bis zu \SI{10,03}{V} mit einer Dauer von ebenfalls \(> \SI{3}{\micro \second}\).
		%
	\section{Unterdrückung von Störspannungen}\label{sec:stoerspannungen}
		Für das Verhalten der Schaltung unter Einfluss einer Störspannung gibt das Datenblatt eine Dämpfung von \(PSRR = \SI{57}{dB}\) ohne
		\(C_{adj}\) und \(PSRR = \SI{64}{dB}\) mit \(C_{adj} = \SI{10}{\micro F}\) an. Testbedingungen zu diesen Angaben sind eine
		Frequenz von \(f = \SI{120}{Hz}\) und einer Eingangsspannung von \SI{10}{V}.\par
		\begin{figure}[h]
			\centering
			\includesvg[inkscapelatex=false, width=.8\textwidth]{LTSpice/plots_schematics/ripple_rejection_PSRR}
			\caption[Wellenform der Ausgangsspannung mit Störspannung]{Wellenform der Ausgangsspannung für \(C_{adj} = \SI{10}{\micro F}\) (blau) und \(C_{adj} = \SI{0}{F}\) (grün).}
			\label{fig:ripple rejection}
		\end{figure}
		Um die simulierte Schaltung dahingehend zu überprüfen wurde eine sinusförmige Störspannung mit einer Amplitude \(\hat{U}_{pp} = 2\) und einer Frequenz gemäß
		Datenblatt von \SI{120}{Hz} um das Eingangsspannungsniveau gelegt. Weiter wird die Schaltung mit \(I_{out} = \SI{500}{mA}\) belastet, was die störungsfreie Ausgangsspannung
		auf \(U_{out} = \SI{5,07244}{V}\) reduziert. \Cref{fig:ripple rejection} sind Spitzenwerte der Ausgangsspannung mit
		\(\hat{U}_{out,0} \approx \SI{5,0725754}{V}\) für \(C_{adj} = \SI{0}{F}\) und \(\hat{U}_{out,10} \approx \SI{5,0725005}{V}\) für \(C_{adj} = \SI{10}{\micro F}\) abzulesen.
		\begin{equation}
			PSRR_i = 20 \cdot log_{10}\left(\frac{2 \cdot \left(\hat{U}_{out,i} - U_{out}\right)}{\hat{U}_{pp}}\right)
		\end{equation}
		errechnen sich damit Dämpfungen von \SI{-77,37}{dB} bzw. \SI{-84,36}{dB}. Zum näheren Vergleich zeigt \cref{fig:bode} ein \textsc{Bode}-Diagram des
		Frequenzverhaltens der Schaltung im Bereich von \SI{100}{Hz} bis \SI{1}{MHz}. Hier fallen die jeweiligen Dämpfungen deutlich höher aus, was auf Ablesefehler
		bei den zur Berechnung herangezogenen Zahlenwerten schließen lässt.
		\begin{figure}[h]
			\centering
			\includesvg[inkscapelatex=false, width=.8\textwidth]{LTSpice/plots_schematics/bode_ripple_rejection}
			\caption[\textsc{Bode}-Diagram der Ausgangsspannung mit überlagerter Störspannung]{\textsc{Bode}-Diagram der Ausgangsspannung mit überlagerter Störspannung im Frequenzbereich
			von \SI{100}{Hz} bis \SI{1}{MHz} für \(C_{adj} = \SI{10}{\micro F}\) (blau) und \(C_{adj} = \SI{0}{F}\) (grün).}
			\label{fig:bode}
		\end{figure}
		%
	\section{Verhalten bei Lastsprüngen}
		Unter Kapitel 7.5 des Datenblattes \textit{Load regulation} werden für Lastsprünge von \SI{10}{mA} auf \SI{1500}{mA} in Anwesenheit des Kondensators \(C_{Adj} = \SI{10}{\micro F}\)
		typische Änderungen der Ausgangsspannung von \SI{0,1}{\percent} oder maximal \SI{25}{mV} angegeben.\par
		\begin{figure}[h]
			\begin{subfigure}[]{\textwidth}
				\centering
				\includesvg[inkscapelatex=false, width=\textwidth]{LTSpice/plots_schematics/load-transient_response}
				\caption[]{Der Verlauf der Ausgangsspannung bei Lastsprüngen (unten) und der zugehörige, rechteckförmige Lastsprung (oben).}
				\label{subfig:load transient response}
			\end{subfigure}
			\vspace{3mm}
			\begin{subfigure}[]{\textwidth}
				\includesvg[inkscapelatex=false, width=\textwidth]{LTSpice/plots_schematics/load-transient_response_deltaU}
				\caption[]{Zoom auf den blau markierten Bereich. Zu sehen ist eine Differenz der Ausgangsspannung im eingeschwungenen Zustand von \(\approx \SI{-4}{mV}\) gegenüber dem
				mit \SI{10}{\micro F} kapazitiv gestützen Adjust-Anschluss.}
				\label{subfig:load transient response deltaU}
			\end{subfigure}
			\caption[Verhalten der Schaltung bei Lastsprüngen]{Verhalten der Schaltung bei Lastsprüngen. (a): Wellenformen der Ausgangsspannung bei verschiedenen Werten für \(C_{Adj}\).
			(b): Zoom auf den blau markierten Bereich in (a).}
		\end{figure}
		\Cref{subfig:load transient response deltaU} zeigt im eingeschwungenen Zustand bei einer Last von \SI{1500}{mA} eine
		Ausgangsspannung von \(\approx \SI{5,0725}{V}\) (untere Kurve). Zur Referenz wurde das Verhalten auch bei Abwesenheit von \(C_{Adj}\) durchgeführt, was
		in einer eingeschwungenen Spannung von \(\approx \SI{5,064}{V}\) resultiert.\par
		Die relativen und absoluten Spannungsänderungen gegenüber der \(U_{out}\) der unbelasteten Schaltung (vgl. \cref{sec:leerlaufspannung}) sind
		\begin{equation}
			\Delta U_{out,10} = U_{out} - U_{out,10} = \SI{5,07525}{V} - \SI{5,0725}{V} = \SI{2,75}{mV} \qquad \Rightarrow \qquad \approx \SI{0,054}{\percent}
		\end{equation}
		bzw.
		\begin{equation}
			\Delta U_{out,0} = U_{out} - U_{out,0} = \SI{5,07525}{V} - \SI{5,064}{V} = \SI{11,25}{mV} \qquad \Rightarrow \qquad \approx \SI{0,222}{\percent}
		\end{equation}
		Beide Werte liegen unterhalb der Angabe für die \textit{typische} Abweichung und deutlich unterhalb der Angabe für die \textit{maximalen} Abweichungen.\par\medskip
		%
		Die dynamischen Ausgangswiderstände ergeben sich so zu
		\begin{align}
			r_{10} = \frac{\Delta U_{out,10}}{\Delta I_{load}} = \frac{\SI{5,0725}{V}}{\SI{1500}{mA} - \SI{10}{mA}} \approx \SI{1,85}{m\ohm} \nonumber \\
			r_{0} = \frac{\Delta U_{out,10}}{\Delta I_{load}} = \frac{\SI{5,064}{V}}{\SI{1500}{mA} - \SI{10}{mA}} \approx \SI{7,55}{m\ohm}
		\end{align}
		%
		\newpage
	\section{Wirkungsgrad des Netzteils}
		Zur Abschätzung des Einflusses der Eingangsspannung auf den Wirkungsgrad der Schaltung wird der Linearregler mit \(R_{load} = \SI{6}{\ohm}\) belastet und
		die Eingansspannung im Intervall \(\SI{10}{V} \leq U_{in} \leq \SI{20}{V}\) variiert. \Cref{fig:efficiency} zeigt den Verlauf einerseits der Verlustleistung
		am LM317 in rot und den Wirkungsgrad der Schaltung in blau.\par
		Zur Berechnung des Wirkungsgrades wurde die Gleichung
		\begin{equation}
			\eta = \frac{P_{out}}{P_{in}}
		\end{equation}
		mit \(P_{out} = U_{loud} \cdot I_{load}\) und \(P_{in} = U_{in} \cdot I_{in}\) heran gezogen.
		\begin{figure}[h]
			\centering
			\includesvg[inkscapelatex=false, width=.9\textwidth]{LTSpice/plots_schematics/efficiency}
			\caption[Verlauf der Verlustleistung und des Wirkungsgrades jeweils als Funktion der Eingangsspannung]{Verlauf der Verlustleistung im Linearspannungsregler (rot) und des
			Wirkungsgrades (blau) jeweils als Funktion der Eingangsspannung im Intervall \(\SI{10}{V} \leq U_{in} \leq \SI{20}{V}\) aufgetragen.}
			\label{fig:efficiency}
		\end{figure}
		Es ist hier zu bemerken, dass bei höheren Eingangsspannungen Maßnahmen ergriffen werden müssen, um die zusätzliche Verlustleistung abführen zu können.